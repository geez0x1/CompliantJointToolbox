%\documentclass[letterpaper, 9 pt, conference]{article}  % Comment this line out if you need a4paper
\documentclass[a4paper,10pt]{cjtdsheet}      % Use this line for a4 paper

\usepackage{lipsum}

\begin{document}
\renewcommand{\arraystretch}{1.2}
\actuatorname{Walk-Man Big Motor (B11)}

\begin{multicols}{2}
\begin{tabularx}{0.95\columnwidth}[c]{p{3cm}lXr}
   \rowcolor{cjtblue}
   \textcolor{white}{\textbf{Mechanical}} 
   & \textcolor{white}{\textbf{Symbol}} 
   & \textcolor{white}{\textbf{Unit}} 
   & \textcolor{white}{\textbf{Value}} 
  \tabularnewline
   %
        Transmission Ratio                  & $\symGearratio$      &  -                    & \valGearratio           \tabularnewline \rowcolor{lightgray}
        Mass                                & $\symMass $          & kg                    & \valMass                \tabularnewline 
        Rotor Inertia                       & $\symInertiarotor$   & $\text{kg}\text{m}^2$ & \valInertiarotor        \tabularnewline \rowcolor{lightgray}
        Gear Inertia                        & $\symInertiagear$    & $\text{kg}\text{m}^2$ & \valInertiagear         \tabularnewline 
        Spring Inertia                      & $\symInertiaspring$  & $\text{kg}\text{m}^2$ & \valInertiaspring       \tabularnewline \rowcolor{lightgray}
        Diameter                            & $\symDiameter$       & mm                    & \valDiameter            \tabularnewline 
        Length                              & $\symActlength$      & mm                    & \valActlength           \tabularnewline \rowcolor{lightgray}
        Mechanical Time \newline Constant   & $\symTmech$          & s                     & \valTmech               \tabularnewline 
        Viscous Friction                    & $\symViscousdamping$ & Nm s/rad              & \valViscousdamping      \tabularnewline \rowcolor{lightgray}
        Coulomb Friction                    & $\symCoulombdamping$ & Nm                    & \valCoulombdamping      \tabularnewline 
%
%
%
    \rowcolor{cjtblue}
    \textcolor{white}{\textbf{Electrical}}   
        & \textcolor{white}{\textbf{Symbol}} 
        & \textcolor{white}{\textbf{Unit}} 
        & \textcolor{white}{\textbf{Value}}
    \tabularnewline
        Winding \newline Resistance        & $\symArmaturresistance$  & $\Omega$          & \valArmaturresistance   \tabularnewline    \rowcolor{lightgray}
        Winding \newline Inductance        & $\symArmatureinductance$ &  mH               & \valArmatureinductance  \tabularnewline
        Electrical \newline Time Constant  & $\symTel$                &  s                & \valTel                 \tabularnewline    \rowcolor{lightgray}
        Torque Constant                    & $\symTorqueconstant$     &  Nm/A             & \valTorqueconstant      \tabularnewline    
        Speed Constant                     & $\symSpeedconstant$      &  Vs/rad           & \valSpeedconstant       \tabularnewline    \rowcolor{lightgray}
%
    \rowcolor{cjtblue}
    \textcolor{white}{\textbf{Thermal}} 
        & \textcolor{white}{\textbf{Symbol}} 
        & \textcolor{white}{\textbf{Unit}} 
        & \textcolor{white}{\textbf{Value}} 
    \tabularnewline
    Max. Temperature                    & $\symTmpWindMax$     &  C          & \valTmpWindMax         \tabularnewline     
    Winding \newline Time Constant      & $\symTthw$           &  s          & \valTthw               \tabularnewline     \rowcolor{lightgray}
    Motor \newline Time Constant        & $\symTthm$           &  s          & \valTthm               \tabularnewline     
\end{tabularx}

%%%
%%%
%%%
\begin{tabularx}{0.95\columnwidth}[c]{p{3cm}lXr}
%
%
%
    Therm. Resistance \newline Winding-Housing & $\symResthermWH$         &  K/W        & \valResthermWH          \tabularnewline     \rowcolor{lightgray}
    Therm. Resistance \newline Housing-Air     & $\symResthermHA$         &  K/W        & \valResthermHA          \tabularnewline     
    \rowcolor{cjtblue}
    \textcolor{white}{\textbf{Rated Operation}} 
        & \textcolor{white}{\textbf{Symbol}} 
        & \textcolor{white}{\textbf{Unit}}
        & \textcolor{white}{\textbf{Value}}
    \tabularnewline
    Rated Voltage                   & $\symRatedvoltage$         & V                & \valRatedvoltage        \tabularnewline     
    Rated Current                   & $\symRatedcurrent$         & A                & \valRatedcurrent        \tabularnewline    \rowcolor{lightgray}
    Rated Torque                    & $\symRatedtorque $         & Nm               & \valRatedtorque         \tabularnewline    
    Rated Speed                     & $\symRatedspeed  $         & rad/s            & \valRatedspeed          \tabularnewline    \rowcolor{lightgray}
    Rated El. Power                 & $\symRatedpowere $         & W                & \valRatedpowere         \tabularnewline    
    Rated Mech. Power               & $\symRatedpowerm $         & W                & \valRatedpowerm         \tabularnewline    \rowcolor{lightgray}
    No Load Current                 & $\symNoloadcurrent$        & A                & \valNoloadcurrent       \tabularnewline    
    No Load Torque                  & $\symNoloadtorque $        & Nm               & \valNoloadtorque        \tabularnewline    \rowcolor{lightgray}
    No Load Speed                   & $\symNoloadspeed  $        & rad/s            & \valNoloadspeed         \tabularnewline    
    Stall Torque                    & $\symStalltorque  $        & Nm               & \valStalltorque         \tabularnewline    \rowcolor{lightgray}
    Starting Current                & $\symStartingcurrent$      & A                & \valStartingcurrent     \tabularnewline    
    Torque Speed \newline Gradient  & $\symSpeedtorquegradient$  & rad/Nms          & \valSpeedtorquegradient \tabularnewline    \rowcolor{lightgray}
%
%
%
    \rowcolor{cjtblue}
    \textcolor{white}{\textbf{Peak Operation}}   
        & \textcolor{white}{\textbf{Symbol}} 
        & \textcolor{white}{\textbf{Unit}} 
        & \textcolor{white}{\textbf{Value}} 
    \tabularnewline
    Peak Current                    & $\symMaxcurrent$ & A                 & \valMaxcurrent     \tabularnewline     
    Peak Torque                     & $\symMaxtorque$  & Nm                & \valMaxtorque      \tabularnewline     \rowcolor{lightgray}
    Peak Speed                      & $\symMaxspeed $  & rad/s             & \valMaxspeed       \tabularnewline     
    Peak El. Power                  & $\symMaxpowere$  & W                 & \valMaxpowere      \tabularnewline     \rowcolor{lightgray}
    Peak Mech. Power                & $\symMaxpowerm$  & W                 & \valMaxpowerm      \tabularnewline 
%
%   Max. Efficiency                 & $\eta_{max}$         & \%                   & \valMaxefficiency      \tabularnewline     \rowcolor{lightgray} 
    \end{tabularx}

\end{multicols}

%\begin{figure}[!H]
%\caption{Speed-Torque Curve of the actuator.}
%\centering
    %\includegraphics{dummy.pdf}
%\end{figure}

\begin{center}
    \captionof{figure}{Speed-Torque Curve of the actuator.}
    \includegraphics{dummy.pdf}
\end{center}

\newpage

\section*{\textcolor{cjtred}{Parameter Definitions}}

\begin{multicols}{2}

\cjtemph{Transmission Ratio:} The actuator typically comprises a transmission mechanism such as a gear box, that amplifies the torque output trading off the deliverable speed. The value is the ratio of the transmission input speed to the obtained output speed. Conversely, it describes the ratio of the output torque over the input torque.

\cjtemph{Mass:} The actuator comprises of three main parts: the motor, the transmission mechanism and the torque sensor. The value is the summed mass of all three components.

\cjtemph{Rotor Inertia:} The rotary inertia of the motor shaft.

\cjtemph{Transmission Inertia:} The rotary inertia of the transmission mechanism.

\cjtemph{Spring Inertia:} The rotary inertia of the deflective element used for torque sensing.

\cjtemph{Diameter:} The maximum diameter of the actuator considering all components: the motor, transmission mechanism and torque sensor.

\cjtemph{Length:} The overall length of the actuator including the motor, transmission mechanism and torque sensor.

\cjtemph{Mechanical Time Constant:} Describes the acceleration of the actuator when a constant voltage is applied, under the condition that the required starting current can be provided. The value is the time for the actuator output to arrive at 63 \% of the steady state velocity for the applied voltage ignoring friction and additional load effects. The mechanical time constant is computed to:
\begin{equation}
\end{equation}

\cjtemph{Viscous Friction:} The transmission ratio...

\cjtemph{Coulomb Friction:} The transmission ratio...

\end{multicols}{2}

\end{document}